\PassOptionsToPackage{unicode}{hyperref}
\documentclass[aspectratio=169]{beamer}

\usetheme{Madrid}
\usecolortheme{default}
\setbeamertemplate{navigation symbols}{}

\usepackage[T1]{fontenc}
\usepackage[utf8]{inputenc}
\usepackage{lmodern}
\usepackage[portuguese]{babel}
\usepackage{graphicx}
\usepackage{booktabs}
\usepackage{tabularx}
\usepackage{xcolor}
\usepackage{listings}
\usepackage{listingsutf8}
\usepackage{ulem}
% Habilita Unicode em metadados (bookmarks etc.)
\hypersetup{unicode=true}

% Cor mais visível para texto monoespaçado inline (\texttt{...})
\definecolor{ttcolor}{RGB}{0,51,153} % azul mais forte
\DeclareTextFontCommand{\textttblue}{\ttfamily\color{ttcolor}}
\DeclareTextFontCommand{\textttwhite}{\ttfamily\color{white}}
\let\texttt\textttblue

% Usar branco para \texttt em títulos de slides (frametitle e framesubtitle)
\makeatletter
\addtobeamertemplate{frametitle}{\begingroup\let\texttt\textttwhite}{\endgroup}
\addtobeamertemplate{framesubtitle}{\begingroup\let\texttt\textttwhite}{\endgroup}
\makeatother

\lstdefinestyle{py}{
  language=Python,
  basicstyle=\ttfamily\small,
  keywordstyle=\color{blue!70!black},
  commentstyle=\color{gray!70!black},
  stringstyle=\color{teal!60!black},
  showstringspaces=false,
  frame=single,
  rulecolor=\color{black!15},
  numbers=left,
  numberstyle=\tiny\color{gray!80!black},
  numbersep=8pt,
  tabsize=2,
  columns=fullflexible,
  keepspaces=true,
  upquote=true,
  breaklines=true,
  breakatwhitespace=true,
  xleftmargin=2em,
  framexleftmargin=2em,
  framexrightmargin=0em
}
\lstset{
  style=py,
  inputencoding=utf8,
  literate=
    {á}{{\'a}}1 {à}{{\`a}}1 {ã}{{\~a}}1 {â}{{\^a}}1
    {é}{{\'e}}1 {ê}{{\^e}}1
    {í}{{\'\i}}1
    {ó}{{\'o}}1 {ô}{{\^o}}1 {õ}{{\~o}}1
    {ú}{{\'u}}1 {ü}{{\"u}}1
    {ç}{{\c c}}1
    {Á}{{\'A}}1 {À}{{\`A}}1 {Ã}{{\~A}}1 {Â}{{\^A}}1
    {É}{{\'E}}1 {Ê}{{\^E}}1
    {Í}{{\'I}}1
    {Ó}{{\'O}}1 {Ô}{{\^O}}1 {Õ}{{\~O}}1
    {Ú}{{\'U}}1 {Ü}{{\"U}}1
    {Ç}{{\c C}}1
}

\title{Pandas: Guia Básico}
\subtitle{Do básico ao essencial}
\author{Claudio Scheer}
\date{\today}

\begin{document}

\begin{frame}
  \titlepage
\end{frame}

\begin{frame}{\texttt{pandas}: o que é?}
  \begin{itemize}
    \item Biblioteca de código aberto para \textbf{análise e manipulação} de dados
    \item Estruturas principais: \textbf{Series} (1D) e \textbf{DataFrame} (2D)
    \item Integração com \texttt{Matplotlib} e o ecossistema Python
    \item Leitura/escrita em vários formatos: CSV, Excel, JSON, SQL
    \item Ferramentas para limpeza, seleção, agrupamento e agregação
  \end{itemize}
\end{frame}

\begin{frame}{\textbf{Índice}}
  \large
  \begin{columns}
    \begin{column}{0.48\textwidth}
      \begin{enumerate}
        \item O que é Pandas?
        \item Instalação e Importação
        \item Estruturas: Series e DataFrame
        \item Operações Básicas
        \item Leitura e Escrita (CSV/Excel/JSON)
        \item Seleção e Filtragem
        \item Limpeza de Dados
      \end{enumerate}
    \end{column}
    \begin{column}{0.48\textwidth}
      \begin{enumerate}
        \setcounter{enumi}{7}
        \item Operações com Dados
        \item Agrupamento e Agregação
        \item Visualização
        \item Merge/Join/Concat
        \item Datas
        \item Dicas e Recursos
        \item Exercícios
      \end{enumerate}
    \end{column}
  \end{columns}
\end{frame}

\begin{frame}[fragile]{Instalação e Importação}
\begin{lstlisting}[language=sh]
# via pip
pip install pandas

# via conda
conda install pandas
\end{lstlisting}
\vspace{0.4em}
\begin{lstlisting}
import pandas as pd
\end{lstlisting}
\end{frame}

\begin{frame}[fragile]{Series (1D)}
  \begin{block}{Ideia}
    Uma \textbf{Series} é como uma coluna de valores com um índice.
  \end{block}
\begin{lstlisting}
# Series simples
numeros = pd.Series([10, 20, 30, 40, 50])
print(numeros)

# índice personalizado
frutas = pd.Series(['Maçã', 'Banana', 'Laranja'], index=['a','b','c'])
print(frutas)

# a partir de dicionário
vendas = pd.Series({'Jan': 1000, 'Fev': 1500, 'Mar': 1200})
print(vendas['Jan'])  # 1000
\end{lstlisting}
\end{frame}

\begin{frame}[fragile]{DataFrame (2D)}
  \begin{block}{Ideia}
    Um \textbf{DataFrame} é como uma planilha: linhas e colunas com rótulos.
  \end{block}
\begin{lstlisting}
dados = {
    'Nome': ['Ana', 'João', 'Maria', 'Pedro'],
    'Idade': [25, 30, 28, 35],
    'Cidade': ['SP', 'RJ', 'BH', 'SP'],
    'Salario': [5000, 6000, 5500, 7000]
}
df = pd.DataFrame(dados)
print(df.head())

# a partir de lista de listas
dados_lista = [['Ana', 25, 'SP'], ['João', 30, 'RJ']]
df2 = pd.DataFrame(dados_lista, columns=['Nome','Idade','Cidade'])
\end{lstlisting}
\end{frame}

\begin{frame}[fragile]{Operações Básicas}
\begin{lstlisting}
df.head()      # primeiras linhas
df.tail()      # últimas linhas
df.info()      # tipos e memória
df.shape       # (linhas, colunas)
df.columns     # nomes das colunas
df.index       # índice das linhas
df.dtypes      # tipos por coluna
\end{lstlisting}
\vspace{0.4em}
\begin{lstlisting}
df_num = pd.DataFrame({
  'Vendas':[100,150,200,180,220],
  'Custos':[ 50, 70, 90, 85,100]
})
df_num.describe()  # contagem, média, desvio, min, quartis, max
df_num.mean()      # médias por coluna
\end{lstlisting}
\end{frame}

\begin{frame}[fragile]{Leitura e Escrita}
\begin{lstlisting}
# leitura
df_csv   = pd.read_csv('arquivo.csv')
df_excel = pd.read_excel('arquivo.xlsx')
df_json  = pd.read_json('arquivo.json')

# escrita
df.to_csv('saida.csv', index=False)
df.to_excel('saida.xlsx', index=False)
df.to_json('saida.json', orient='records')
\end{lstlisting}
\end{frame}

\begin{frame}[fragile]{Seleção e Filtragem}
\begin{lstlisting}
# colunas
idades = df['Idade']
subconjunto = df[['Nome','Salario']]

# linhas por posição / rótulo
df.iloc[0]       # primeira linha
df.loc[0:2]      # linhas 0 a 2 (inclusivo)

# por condição
maiores_30 = df[df['Idade'] > 30]
sp_jovens = df[(df['Cidade'] == 'SP') & (df['Idade'] < 30)]

# avançado
df.loc[0:2, ['Nome','Idade']]
df.query('Idade > 25 and Salario < 6000')
\end{lstlisting}
\end{frame}

\begin{frame}[fragile]{Limpeza de Dados}
\begin{lstlisting}
df_faltantes = pd.DataFrame({
  'A':[1,2,float('nan'),4],
  'B':[5,float('nan'),float('nan'),8]
})

df_faltantes.isnull().sum()   # contar NaN por coluna
df_faltantes.dropna()         # remover linhas com NaN
df_faltantes.dropna(axis=1)   # remover colunas com NaN

# preencher
df_faltantes.fillna(0)
df_faltantes.fillna(df_faltantes.mean(numeric_only=True))
df_faltantes.fillna(method='ffill')
\end{lstlisting}
\end{frame}

\begin{frame}[fragile]{Duplicatas, Renomear e Tipos}
\begin{lstlisting}
df_dup = pd.DataFrame({'Nome':['Ana','João','Ana'], 'Idade':[25,30,25]})
df_dup.duplicated()
df_dup.drop_duplicates()

# renomear
df_renomeado = df.rename(columns={'Nome':'NomeCompleto', 'Idade':'Anos'})

# tipos
df['Idade'] = df['Idade'].astype(float)
df['Salario'] = pd.to_numeric(df['Salario'], errors='coerce')
\end{lstlisting}
\end{frame}

\begin{frame}[fragile]{Operações com Dados}
\begin{lstlisting}
# novas colunas
df['Bonus'] = df['Salario'] * 0.10
df['SalarioTotal'] = df['Salario'] + df['Bonus']

# coluna condicional
df['Senioridade'] = df['Idade'].apply(lambda x: 'Sênior' if x >= 30 else 'Júnior')

# remover coluna
df = df.drop('Bonus', axis=1)  # ou: del df['Bonus']
\end{lstlisting}
\end{frame}

\begin{frame}[fragile]{Ordenação}
\begin{lstlisting}
# ordenação
df.sort_values('Salario')
df.sort_values('Salario', ascending=False)
df.sort_values(['Cidade','Idade'], ascending=[True, False])
df.sort_index()
\end{lstlisting}
\end{frame}

\begin{frame}[fragile]{Agrupamento e Agregação}
\begin{lstlisting}
df_vendas = pd.DataFrame({
  'Vendedor':['Ana','João','Ana','João','Maria','Maria'],
  'Produto':['A','A','B','B','A','B'],
  'Quantidade':[10,15,20,25,30,35],
  'Valor':[100,150,200,250,300,350]
})

g = df_vendas.groupby('Vendedor')
g.sum()
g.mean()

# múltiplas agregações
agg = df_vendas.groupby('Vendedor').agg({'Quantidade':'sum', 'Valor':['mean','sum']})
\end{lstlisting}
\end{frame}

\begin{frame}[fragile]{Pivot Table}
\begin{lstlisting}
pivot = df_vendas.pivot_table(
  values='Valor', index='Vendedor', columns='Produto',
  aggfunc='sum', fill_value=0
)
print(pivot)
\end{lstlisting}
\end{frame}

\begin{frame}[fragile]{Visualização (básico)}
\begin{lstlisting}
import matplotlib.pyplot as plt

df_plot = pd.DataFrame({
  'Mes':['Jan','Fev','Mar','Abr','Mai'],
  'Vendas':[100,120,140,110,160],
  'Custos':[ 80, 90,100, 85,110]
})

df_plot.plot(x='Mes', y='Vendas', kind='line')
plt.title('Vendas Mensais')
plt.show()

df_plot.plot(x='Mes', y=['Vendas','Custos'], kind='bar')
plt.title('Vendas vs Custos')
plt.show()
\end{lstlisting}
\end{frame}

\begin{frame}[fragile]{Merge, Join e Concat}
\begin{lstlisting}
# concatenação
df1 = pd.DataFrame({'A':[1,2], 'B':[3,4]})
df2 = pd.DataFrame({'A':[5,6], 'B':[7,8]})
pd.concat([df1, df2], ignore_index=True)

# junção (merge)
esq = pd.DataFrame({'ID':[1,2,3], 'Nome':['Ana','João','Maria']})
dir = pd.DataFrame({'ID':[1,2,4], 'Salario':[5000,6000,7000]})

pd.merge(esq, dir, on='ID')            # inner
pd.merge(esq, dir, on='ID', how='left')
pd.merge(esq, dir, on='ID', how='outer')
\end{lstlisting}
\end{frame}

\begin{frame}[fragile]{Trabalhando com Datas}
\begin{lstlisting}
datas = pd.date_range('2024-01-01', periods=10, freq='D')
df_tempo = pd.DataFrame({'Data': datas, 'Valor': range(100, 110)})
df_tempo.set_index('Data', inplace=True)

df_tempo.resample('W').mean()     # média semanal
df_tempo.rolling(window=3).mean() # média móvel

df_tempo['Ano'] = df_tempo.index.year
df_tempo['Mes'] = df_tempo.index.month
\end{lstlisting}
\end{frame}

\begin{frame}[fragile]{Dicas Rápidas}
  \begin{itemize}
    \item Prefira \textbf{vetorização} a loops explícitos
    \item Use \texttt{query()} para filtros legíveis
    \item Para arquivos grandes, leia em \textit{chunks}
    \item Use tipo \texttt{category} para colunas com poucos valores distintos
  \end{itemize}
\vspace{0.4em}
\begin{lstlisting}
# filtro claro (bom)
df.query('col1 > 10 and col2 < 20')
\end{lstlisting}
\end{frame}

\begin{frame}{Recursos Úteis}
  \small
  \begin{itemize}
    \item Documentação Oficial: \href{https://pandas.pydata.org}{pandas.pydata.org}
    \item 10 Minutes to Pandas: \href{https://pandas.pydata.org/pandas-docs/stable/user_guide/10min.html}{tutorial rápido}
    \item Cheat Sheet: \href{https://pandas.pydata.org/Pandas_Cheat_Sheet.pdf}{referência rápida}
    \item Kaggle Learn: \href{https://www.kaggle.com/learn/pandas}{cursos práticos}
  \end{itemize}
\end{frame}

\begin{frame}{Exercícios Práticos}
  \begin{enumerate}
    \item Crie um DataFrame com 5 produtos (nome, preço, quantidade) e calcule o valor total do estoque.
    \item Leia um CSV, limpe valores faltantes, agrupe por categoria e crie um gráfico de barras.
    \item Faça merge de dois DataFrames, crie uma tabela dinâmica (pivot) e exporte para Excel.
  \end{enumerate}
\end{frame}

\begin{frame}
  \centering
  \Large Obrigado! \\
  \vspace{0.5em}
  \normalsize Perguntas?
\end{frame}

\end{document}